\documentclass[12pt]{article}
\usepackage{amsmath, amssymb, bm}

\title{ECE 712: Mini Assignment 3}
\author{Dai, Jiatao}
\date{Oct 25, 2025}

\begin{document}
\maketitle

\section*{Problem}
Let the singular value decomposition (SVD) of a matrix $A \in \mathbb{R}^{m \times n}$ be
\[
A = U \Sigma V^{T},
\]
where
\[
U = \begin{bmatrix} U_1 & U_2 \end{bmatrix}, \quad
V = \begin{bmatrix} V_1 & V_2 \end{bmatrix}, \quad
\Sigma =
\begin{bmatrix}
\Sigma_r & 0 \\
0 & \Sigma_2
\end{bmatrix},
\]
and $\Sigma_r = \mathrm{diag}(\sigma_1,\dots,\sigma_r)$ contains the $r$ largest singular values. The truncated rank-$r$ approximation of $A$ is defined as
\[
A_r = U_1 \Sigma_r V_1^{T}.
\]

\begin{itemize}
    \item Show that $A_r = P_c A$, where $P_c = U_1 U_1^T$ is a projection matrix, and identify the corresponding subspace.
    \item Show that $A_r^T = P_r A^T$, where $P_r = V_1 V_1^T$ is a projection matrix, and identify the corresponding subspace.
\end{itemize}

\section*{Solution}

\subsection*{SVD and truncated approximation}
From the SVD, we expand $A$ as
\[
A = U \Sigma V^T =
\begin{bmatrix} U_1 & U_2 \end{bmatrix}
\begin{bmatrix}
\Sigma_r & 0 \\
0 & \Sigma_2
\end{bmatrix}
\begin{bmatrix} V_1^T \\ V_2^T \end{bmatrix}.
\]

The truncated rank-$r$ approximation is
\[
A_r = U_1 \Sigma_r V_1^T.
\]

\subsection*{1) Proof that $A_r = P_c A$}
Define the projection matrix
\[
P_c = U_1 U_1^T.
\]
Then
\[
\begin{aligned}
P_c A
&= U_1 U_1^T \, U \Sigma V^T \\
&= U_1
\begin{bmatrix} I & 0 \end{bmatrix}
\begin{bmatrix}
\Sigma_r & 0 \\
0 & \Sigma_2
\end{bmatrix}
\begin{bmatrix} V_1^T \\ V_2^T \end{bmatrix} \\
&= U_1 \Sigma_r V_1^T \\
&= A_r.
\end{aligned}
\]

\noindent
Thus, $A_r$ is obtained by projecting $A$ onto the subspace
\[
\mathcal{S}_c = \mathrm{span}(U_1),
\]
which is called \textbf{the $r$-dimensional subspace defined by $U_1$, i.e. a subspace of R(A)}.
If $r = \mathrm{rank}(A)$, then $\mathcal{S}_c = \mathcal{R}(A)$, the column space of $A$.

\subsection*{2) Proof that $A_r^T = P_r A^T$}
Define the projection matrix
\[
P_r = V_1 V_1^T.
\]
Using transpose,
\[
A_r^T = (U_1 \Sigma_r V_1^T)^T = V_1 \Sigma_r U_1^T.
\]
Now compute:
\[
\begin{aligned}
P_r A^T
&= V_1 V_1^T (V \Sigma^T U^T) \\
&= V_1
\begin{bmatrix} I & 0 \end{bmatrix}
\Sigma_r
\begin{bmatrix} U_1^T \\ U_2^T \end{bmatrix} \\
&= V_1 \Sigma_r U_1^T \\
&= A_r^T.
\end{aligned}
\]

\noindent
Thus, $A_r^T$ is obtained by projecting $A^T$ onto the subspace
\[
\mathcal{S}_r = \mathrm{span}(V_1),
\]
which is like question 1, called \textbf{the $r$-dimensional subspace defined by $V_1$, i.e.\ a subspace of $\mathcal{R}(A^T)$}.
If $r = \mathrm{rank}(A)$, then $\mathcal{S}_r = \mathcal{R}(A^T) = \mathcal{N}(A)^\perp$.

\end{document}